% Fri 7 Oct 18:57

It is common knowledge that malicious software is a problem in computing.
Malicious actors are often motivated by financial and political gains.

Malware can be described as a piece of software that is malicious,
malicious meaning it does things that are bad,
this can include exfiltration of data from a device,
creating damage such as encrypting data and holding the key as ransom.

Malware authors work hard to avoid their work being attributed back to themselves,
adopting strategies such as polymorphic code (code that can change itself during execution),
anti-debugging (changing behaviour or not running at all),
code obfuscation (making code hard to reverse engineer or decrypting the real code on execution).

Malware is commonly packed with a tool called UPX, which allows code to unpack itself on start,
making it difficult to analysis the code, until it is executing and you can dump the real code from memory.

What can be classified as malware is hard to define, false positives is something that plagues modern anti-malware solutions,
sometimes applications used by millions of people can exhibit behaviours comparable to malware,
such as retrieving your contacts list on your smart phone or retrieving a list of your installed applications.

A common technique modern anti-malware solution employ is heuristic analysis,
this is scanning executable file formats for certain known arifacts that other known malware have.
This is not enough as I have explained above, modern malware try to hide their true nature from scanners,
now dynamic heuristic analysis is needed, we can observe what the application is doing while it is running, what resources it is requesting,
a ransomware for example would be opening many files on the computer, and replacing them with encrypted modified versions.

In Ireland in 2021 the Health Service Executive was compromised with the use of a excel spreadsheet document,
this served as the entry point for the Conti ransomeware, this could have been avoided if the excel document was opened in a sandboxed environment mimicking the target computer,
as soon as the ransomware started showing malware signature behaviour it could have been classed as malicious and then be quaratined.

The goal of my project is to create a platform that can facilitate malware detection using dynamic heuristic analysis.
I will be developing an application that will run on the computer natively, it will be a GUI application


% Fri 7 Oct 18:57
\chapter{Introduction}
% What is it about? Is it at the right level (8)?
% Is the scope correct?
% Do not assume that the reader knows anything about the domain.
% Why should a reader care or be interested?

\section{Abstract}
The word malware is derived from the words "malicious" and "software".
It is used to describe software that can perform harmful actions.
These actions include the destruction of data,
the exfiltration of sensitive information from a computer or
holding it for ransom (meaning it encrypts files with a key and holds the key for ransom).
Any of these actions can have significant consequences for the victim.
Malware is a constant threat to organizations and individuals,
as malware authors are getting more and more sophisticated,
they've begun using new techniques that avoid detection from traditional anti-malware solutions.

For the Applied Project and Minor Dissertation module,
I have been tasked with creating a significant software project.
I have decided to create a platform that can detect malware based on its behaviour,
using dynamic heuristic analysis, which involves executing files
in a sandboxed environment, and observing what the file does during its execution,
then comparing its behaviour to known malicious actions.
Its behaviour means how the software interacts with the operating system,
examples of this include reading and writing files.
A sandbox is a method of isolating potentially harmful processes on your computer,
limiting the resources it can access this being sensitive files for example.

It's easy to bypass traditional anti-malware scanning techniques,
such as static analysis (this is looking for known patterns of strings or bytes in a file),
Avoiding detection from the static analysis can be done using tools such as UPX.
UPX has the ability to compress executable files,
hiding artefacts which would be used to identify malware. \cite{upx}
Disguising behaviour during execution is a lot harder than simply hiding artefacts in a file.
That's why this method is much more effective at detecting malware.

This project will be targeting Microsoft Windows 10/11,
it will feature a GUI application that can be installed,
allowing users to automatically scan downloaded files in a sandboxed environment.
Due to the dynamic nature of the platform (requiring files to be executed),
worker computers will need to be set up to handle scanning tasks.
Administrators will be able to perform administrative
tasks using a dashboard in their browser.
These tasks will include adding and removing workers,
and adding malicious patterns to the database.

The novel aspect of this project is the use of dynamic heuristic analysis,
this is an unconventional approach to malware detection,
and at times can be much more effective than static analysis.
This project will bundle everything together into a single platform,
allowing users without prior knowledge of the field to use it.
As soon as the installer is finished installing,
the user can upload files to be scanned immediately.

The platform is designed to be used by corporations,
it can be installed on all the computers on the network.
When a user downloads a file they will be prompted to upload it to the platform for analysis.
The user will see a real-time analysis of the file,
if the file is deemed malicious the user will be prompted to quarantine the file.


% % % % % % % % % % % % % % % % % % % % % % % %
% % % % % % % % % % % % % % % % % % % % % % % %
% % % % % % % % % % % % % % % % % % % % % % % %
% % % % % % % % % % % % % % % % % % % % % % % %


\section{Project Overview}
There will be three main sub-projects included in the platform,
these being the client, server and sandbox.
The dashboard is closely integrated with the server,
and although it is found in the server project,
it is actually running on the client side during execution.

The client is a native graphical user interface (GUI) application,
designed for Microsoft Windows 10/11.

To utilize the platform, users must install the client on their computer(s),
once installed it will automatically launch upon system boot.

When a user downloads a file (such as downloading a file using their web browser),
they will be prompted to upload it to be scanned.
The scanning process provides real-time updates on the status of the file,
including a log box displaying the currently executing task and a progress bar for scan progress.

The client also incorporates a quarantine feature, safeguarding malicious files in a secure location,
with an option for users to restore files to their original location.

The server project is responsible for establishing communication between the client and the sandbox,
while also offering an administrator dashboard for carrying out administrative tasks.

The administrator will be able to view statistics by looking at piecharts
seeing the ratio of malicious to clean scans,
and remove workers, provision new clients and add malicious patterns.

The sandbox is responsible for running malware in a secure environment,
executing files, and recording system call(s) for further analysis.

It will also use static analysis as a fallback,
the static analysis differs from my approach, being that it doesn't actually scan the file
and just looks for patterns of text and bytes in the file.

% Set out the objectives of the project clearly.
% – You will have to address each of these in the evaluation / conclusion.
% – The metrics by which success or failure is measured.

\section{Objectives}
The objectives of this project are to:
\begin{itemize}
\item Develop a robust and reliable software platform capable of detecting
      malicious software based on its behaviour and quarantining it when necessary.
\item Ensure the platform's stability and maintainability by producing clean,
      readable and maintainable code.
\item Create a native GUI application for Microsoft Windows 10/11
      that provides users with a simple and efficient means to
      scan files and monitor the scanning process.
\item Build a user-friendly dashboard for administrators to manage
      the platform's administrative tasks effectively.
\item Establish a server project that facilitates communication between
      clients and the sandbox using RESTful APIs.
\item Create a secure platform that leverages state-of-the-art security
      measures to safeguard against unauthorized access.
\item Produce code that adheres to good design principles,
      enabling easier maintenance and modification.
\end{itemize}

\section{Sub-projects}
\subsection{Client}
The source code for the client can be accessed in the \texttt{client/src} directory,
implemented in Python language, utilizing the Tkinter library to create the graphical user interface,
and using a threading library called "tkthread",
to allow the application to do other tasks, without locking up the application.

The client application comprises three sections: "Main Page" (Caladium), "Quarantine" and "Preferences."

The "Main Page" features a text box displaying the current directory being scanned,
along with a manual scanning button.

Since the execution of Python code requires Python to be installed,
PyInstaller is used to bundle all necessary components to run on any computer.

The problem is, that PyInstaller does not generate a single executable file,
but rather outputs a directory of files.

I am using the "iexpress" tool included in Windows can
create an installer that produces a single executable file.
It will be installed in the \texttt{Program Files} directory of the user's computer.

The "Preferences" page offers an "Uninstall" button,
providing users with the option to remove the client application.

\subsection{Dashboard}
The code of the dashboard will be found in the \texttt{server/src/static} directory.
\footnote{The dashboard may appear as if it is part of the server
but it is actually on the client side, as it runs on the user's computer.}

Administrators will be able to login into the dashboard and perform administrative tasks.
They will also have the ability to change their password and new workers to the system.
The dashboard will be written in JavaScript, it will be a single-page application.

Upon opening the dashboard you will be presented with a login page,
prompting you to enter a username and password, the default for these is "root" and "root".
The password is changeable using the preferences menu.

The dashboard will feature four different types of list pages,
they are clients, patterns, tasks and worker pages.

\subsection{Server}
The code of the server will be found in the \texttt{server} directory.
The server will be responsible for bridging communications between the client and the sandbox.
The server will also be written in Python 3, using the Flask micro-framework for handling HTTP requests.

Data will be persisted using CouchDB since the server is using
python will be using the pycouchdb library to communicate with the
CouchDB instance.

This includes provisioning clients and adding new detection patterns for the dynamic analysis

\subsection{Sandbox}
The code of the sandbox can be found in the \texttt{sandbox/src} directory.
The main part is written in Racket (a functional programming language),

Since this platform is targeting malware that is designed
to run on Microsoft Windows, the sandbox must run on Microsoft Windows.

It makes use of Sandboxie to isolate the malware from the host system,
a tool called Process Monitor to log system calls.

It will also use ClamAV to do static analysis as a fallback, in the case of the dynamic analysis failing.

\subsection{Bot \& Docs}
The platform also features a promotional page,
which supplies new users with an overview of the project with a download button
and screenshots of the platform with descriptions.
When you click the download button it queries the GitHub API to fetch the latest
GitHub release for the client.

There is also a chatbot called CaladiumBot source code can be found in the 
\texttt{bot} directory, it makes use of GPT 3.5 model. \cite{openai}
The bot can answer questions about the platform,
it has 

% Briefly list each chapter / section and provide a brief description of what each section contains.
% – List the resource URL (GitHub address) for the project and provide
% a brief list and description of the main elements at the URL.

\section{Document Overview}
\begin{itemize}
  \item The "Methodology" chapter will cover the way I developed the project,
  including my use of the Kanban board and using Git for source control.
  \item "Technology Review" contains a run down on my initial research on selecting
  technologies for the project, and comparing the alternatives.
  \item "System Design" will explain how I structured the platform,
  and the various design patterns I adopted.
  \item The "System Evaluation" will be me evaluating the system,
  talking through its good parts and potential areas for improvement.
  \item I conclude with the "Conclusion" chapter.
\end{itemize}

The GitHub repository for the project can be found at the URL: \\
\href{https://github.com/G00378925/caladium}{https://github.com/G00378925/caladium}
% Fri 7 Oct 18:57
\chapter{Introduction}
% What is it about? Is it at the right level (8)?
% Is the scope correct?
% Do not assume that the reader knows anything about the domain.
% Why should a reader care or be interested?

\section{Abstract}
The word malware is derived from the words "malicious" and "software".
It is used to describe software that can perform harmful actions.
These actions include the destruction of data,
the exfiltration of sensitive information from a computer or
holding it for ransom (meaning it encrypts files with a key and holds the key for ransom).
Any of these actions can have significant consequences for the victim.
Malware is a constant threat to organizations and individuals.
as malware authors are getting more and more sophisticated,
they've begun using new techniques that avoid detection from traditional anti-malware solutions.

For the Applied Project and Minor Dissertation module,
I have been tasked with creating a significant software project.
I have decided to create a platform that can detect malware based on its behaviour.
using dynamic heuristic analysis, which involves executing files in a sandboxed environment,
and observing what the file does during its execution,
then comparing its behaviour to known malicious actions.
Its behaviour meaning how the software interacts with the operating system,
examples of this include reading and writing files.

It's easy to bypass traditional anti-malware scanning techniques,
such as static analysis (this is looking for known patterns of strings or bytes in a file),
Using tools such as UPX, is made easier than ever before.
Disguising behaviour during execution is a lot harder than simply hiding artefacts in a file.
That's why this method is much more effective at detecting malware.

This project will be targeting Microsoft Windows 10/11,
it will feature a GUI application that can be installed,
allowing users to automatically scan downloaded files in a sandboxed environment.
Due to the dynamic nature of the platform (requiring files to be executed),
worker computers will need to be set up to handle scanning tasks.
Administrators will be able to perform administrative tasks using a dashboard in the browser.
These tasks will include adding and removing workers,
and adding malicious patterns to the database,

The novel aspect of this project is the use of dynamic heuristic analysis,
this is an unconventional approach to malware detection,
and at times can be much more effective than static analysis.
This project will bundle everything together into a single platform,
allowing users without prior knowledge of the field to use it.
As soon as the installer is finished installing,
the user can upload files to be scanned immediately.

The platform is designed to be used by corporations,
it can be installed on all the computers on the network.
When a user downloads a file they will be prompted to upload it to the platform for analysis.
The user will see a real-time analysis of the file,
if the file is deemed malicious the user will be prompted to quarantine the file.


% % % % % % % % % % % % % % % % % % % % % % % %
% % % % % % % % % % % % % % % % % % % % % % % %
% % % % % % % % % % % % % % % % % % % % % % % %
% % % % % % % % % % % % % % % % % % % % % % % %


\section{Project Overview}
The platform will feature three sub-projects, which will be the client, server, and sandbox.

The client will be a native GUI application targeted at Microsoft Windows.
Users will have to install this on their computer,
it will automatically start on system boot.

When the user downloads a file they will be prompted to uploaded it to be scanned.
During the scanning process the user will be given real-time updates on the status of the file.
This will include a log box of currently executing task, and a progress bar of the scan progress.

The client will feature a quarantine, locking away malicious files into a safe location,
the user will also have the option of restoring the file to its original location.

The server will be responsible for bridging communications between the client and the sandbox.

The server will also feature a dashboard allowing administrators to perform administrative tasks.

The sandbox will be responsible for running the malware in a sandboxed environment,
executing the files and logging the system calls.

% Set out the objectives of the project clearly.
% – You will have to address each of these in the evaluation / conclusion.
% – The metrics by which success or failure is measured.

\section{Objectives}
The objectives of this project are as follows:
\begin{itemize}
    \item Develop a software platform capable of detecting malicious software based on its behavior.
    \item The platform must be able to detect malware that is designed to run on Microsoft Windows.
\end{itemize}

\subsection{Client}
The code of the client can be found in the \texttt{client} directory.

It is written in Python, using the Tkinter library to create the GUI.

The client will feature 3 pages, Main Page (Caladium), Quarantine and Preferences.

As running Python code, requires Python to be installed,
I am using a tool called PyInstaller to bundle in everything it needs to run on any computer.

Unforuntely PyInstaller doesn't output a single executable file,
it outputs a directory of files, fortunely I have a tool called iexpress that can create an installer,
and that outputs a single executable.

Users will have the option of uninstalling the client, by using the "Unistall" button in the preferences page.

\subsection{Server}
The code of the server will be found in the \texttt{server} directory.

The server will be responsible for bridging communications between the client and the sandbox.
The server will also be written in Python 3, using the Flask micro-framework for handling HTTP requests.

The platform will also feature a dashboard allowing administrators to configure the platform,
This includes provisioning clients and adding new detection patterns.

\subsubsection{Dashboard}
The code of the dashboard will be found in the \texttt{server/src/static} directory.

Administrators will be able to login into the dashboard and perform administrative tasks.
The dashboard will be written in JavaScript, it will be a single-page application.



\subsection{Sandbox}
The code of the sandbox can be found in the \texttt{sandbox} directory.

The main part is written in Racket (a functional programming language),


Since this platform is targetting malware that is designed to run on Microsoft Windows,
the sandbox must run on Microsoft Windows.

It make use of Sandboxie to isolate the malware from the host system,
a tool called Process Monitor to log system calls,





% Briefly list each chapter / section and provide a brief description of what each section contains.
% – List the resource URL (GitHub address) for the project and provide
% a brief list and description of the main elements at the URL.



% % % % % % % % % % % % % % % % % % % % % % % % % % % % % % % % 

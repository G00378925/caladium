\chapter{System Evaluation}
% Evaluate your project against the objectives set out in the introduction.
% This chapter should present results if applicable and discuss the strengths and weaknesses of your system.
% This is a clear opportunity for you to demonstrate your critical thinking in relation to the project. 

\section{Static Analysis}
The main goal of the project was to develop a platform capable of detecting malicious software based on its behaviour.
This was done by executing the software in a sandboxed environment, and logging the system calls.
And then comparing the system calls to a list of malicious patterns.
When I searched for databases on the internet of known malicious patterns,
I didn't find many results, so I opted to include static analysis aswell as a fallback.

As static analysis is a tried and true method of detecting malware,
there is a plethora of definitions for malicious patterns.

\section{Scalling}
The server is capable of handling multiple clients, but there is only one instance of the server running.
This means in the event of the server crashing, clients will not be able to scan files.

The platform can handle muliple scans at the same time, as multiple workers are supported.

\section{Time Complexity}
When a file was scanned 

\section{Security}
To get into the dashboard you must login with a username and password.
The password is hashed using the SHA-256 algorithm.
Meaning if the server is compromised, the admins password won't be readable.

When you login you are given a session token this is stored in the browser's local storage.
All RESTful API calls require the session token to be sent with the request.

\section{Working with Tables}
Table \ref{table:HexToBin} can be referenced with the label given to the table, i.e. \textbf{\textbackslash{}ref\{table:HexToBin\}}. Note that \LaTeX will place the table wherever it deems fit. Don't bother trying to change where a table or figure is placed until your document is ready for final layout.

\begin{table}
    \begin{tabular}{p{2cm}|p{2cm}|p{2cm}|p{2cm}|p{2cm}|p{2cm}}
        \hline
        \multicolumn{6}{|c|}{Hexadecimal to Binary} \\
        \hline
        Hex & Binary 2 & Hex & Binary & Hex & Binary\\
        \hline
        \hline
        1 & 00000001 & B & 00001011 & 15 & 00010101 \\
        2 & 00000010 & C & 00001100 & 16 & 00010110 \\
        3 & 00000011 & D & 00001101 & 17 & 00010111 \\
        4 & 00000100 & E & 00001110 & 18 & 00011000 \\
        5 & 00000101 & F & 00001111 & 19 & 00011001 \\
        6 & 00000110 & 10 & 00010000 & 1A & 00011010 \\
        7 & 00000111 & 11 & 00010001 & 1B & 00011011 \\
        8 & 00001000 & 12 & 00010010 & 1C & 00011100 \\
        9 & 00001001 & 13 & 00010011 & 1D & 00011101 \\
        A & 00001010 & 14 & 00010100 & 1E & 00011110 \\
        \hline
    \end{tabular}
    \caption{Conversion from Hexadecimal to Binary}
    \label{table:HexToBin}
\end{table}
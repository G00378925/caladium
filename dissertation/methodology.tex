\chapter{Methodology}
% Describe the way you went about your project. Was your approach to the problem valid?
% You need to discuss both your software development methodology and your research methodology.

This chapter will be discussing the methodology I used whilst developing my project
and the research I did to help me develop my project.

% % % % % % % % % % % % % % % % % % % % % % % %
% % % % % % % % % % % % % % % % % % % % % % % %
% % % % % % % % % % % % % % % % % % % % % % % %
% % % % % % % % % % % % % % % % % % % % % % % %

\section{Research}
When I figured out what I wanted to do my project on (that being malware detection),
I had to research technologies that would aid with this,
so I began researching tools that I would use to log the system calls on Windows,
I initially was thinking of writing the logging software myself,
but I then concluded that this would take too much time and
I would end up not fulfilling my initial objectives of creating an entire platform,
so then I found a piece of software called "Process Monitor" that could do this.

In the beginning, I decided to break the project into three parts,
the client, server, and sandbox.

\subsection{Malware detection}
I researched ways of identifying malware based on its behaviour,
I found out about a tool called Process Monitor, which allows you to log all
the system calls on a system.

You can tell Process Monitor to output a log file, with all the system calls.

I found out early on that it didn't allow you to filter only a
specific process and its children, so I created a Python script, to filter processes
containing the name of the executable file being run in the sandbox.

\subsection{Client Installer}
Users need to be able to install the client on their computers,
so I researched tools that could generate an installer for Windows.

\subsection{Database}
I needed to be able to persist data for the server,
I decided early on I didn't want to use SQL due to the
difficulty having to write the queries and I had already used MongoDB in a previous project.
I opted for CouchDB

% % % % % % % % % % % % % % % % % % % % % % % %
% % % % % % % % % % % % % % % % % % % % % % % %
% % % % % % % % % % % % % % % % % % % % % % % %
% % % % % % % % % % % % % % % % % % % % % % % %

\section{Development}

Text editors and git

% % % % % % % % % % % % % % % % % % % % % % % %
% % % % % % % % % % % % % % % % % % % % % % % %
% % % % % % % % % % % % % % % % % % % % % % % %
% % % % % % % % % % % % % % % % % % % % % % % %

\section{Agile Methodology}
When I began the project, I didn't fully understand everything I needed to do,
so I opted for an agile methodology.

Agile allows developers to adapt to changing requirements,
and allows for more flexibility in the development process.

Developers can then run a suite of tests on their code,
and if it passes all the tests, it can be deployed to production.

% Agile / incremental and iterative approach to development.
Kanban Board
I decided I wanted to break up all the tasks into user stories,
I was initially going to use Jira but I then found out that
GitHub supported the kanban board, this was
Throughout the development, I identified tasks that needed to be done.

I created three columns in the kanban board, they being "Todo", "In Progress" and "Done".

When I identified tasks to be done I placed them in the "Todo" column,
and the tasks I was currently working on I had in the "In Progress"

I moved the story to the In Progress column when I started working on it.
When I finished working on the story, I moved it to the Done column.

% % % % % % % % % % % % % % % % % % % % % % % %
% % % % % % % % % % % % % % % % % % % % % % % %
% % % % % % % % % % % % % % % % % % % % % % % %
% % % % % % % % % % % % % % % % % % % % % % % %

\section{Time Management}
Finding time to work on the project was a challenge,
I had other modules to work on as well as my dissertation.

Most of my work was done during the weekends,
when I thought of features to add I would add them to the kanban board.

% % % % % % % % % % % % % % % % % % % % % % % %
% % % % % % % % % % % % % % % % % % % % % % % %
% % % % % % % % % % % % % % % % % % % % % % % %
% % % % % % % % % % % % % % % % % % % % % % % %

\section{Version Control}
Git was used for version control, I used GitHub to host the repository.
I created a directory for each sub-project, the client, the server, and the sandbox.

When writing the dissertation I used Overleaf,
and pushed the changes to the GitHub repository into the \texttt{dissertation} directory.


% % % % % % % % % % % % % % % % % % % % % % % %
% % % % % % % % % % % % % % % % % % % % % % % %
% % % % % % % % % % % % % % % % % % % % % % % %
% % % % % % % % % % % % % % % % % % % % % % % %

\section{Software Testing}
% What about validation and testing?
I used test-driven approach (TDD) to develop the server endpoints.
I used the "unittest" Python library to write unit tests for the server endpoints.

I used the tests as a specification for the server endpoints.
If the tests failed I wasn't implementing the endpoints correctly.

I could run the tests and it would verify if I had broken any of the endpoints.

% % % % % % % % % % % % % % % % % % % % % % % %
% % % % % % % % % % % % % % % % % % % % % % % %
% % % % % % % % % % % % % % % % % % % % % % % %
% % % % % % % % % % % % % % % % % % % % % % % %

\section{Continuous Integration and Deployment}
When I pushed code to the GitHub repository,
I wanted a suite of tests to be run automatically.

So I made a GitHub action that can be found at
\texttt{.github/workflows/run\_tests.yml}.
It then runs the tests found in the \texttt{server/tests} directory.

The server was deployed to Microsoft Azure,
the reproducible environment was created using Docker.

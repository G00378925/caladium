\chapter{Methodology}
% Describe the way you went about your project. Was your approach to the problem valid?
% You need to discuss both your software development methodology and your research methodology.

This chapter will be discussing the methodology I used whilst developing my project

When I figured out what I wanted to do my project on (that being malware detection),
I had to research technologies that would aid with this, so I began researching tools that I would use to log the system calls on Windows,
I initially was thinking of writing the logging software myself, but I then concluded that this would take too much time and I would end up not fulfilling my initial objectives of creating an entire platform, so then I found a piece of software called "Process Monitor" that could do this.

In the beginning, I decided to break the project into three parts, the client, server, and sandbox.

I used test-driven approach (TDD) to develop the server endpoints.
I used the "unittest" python library to write unit tests for the server endpoints.

I used the tests as a specification for the server endpoints.
If the tests failed I wasn't implementing the endpoints correctly

I decided I wanted to break up all the tasks into user stories, I was initially going to use Jira but I then found out that GitHub supported the kanban board, this was
Throughout the development, I identified tasks that needed to be done.

I created three columns in the kanban board, they being "Todo", "In Progress" and "Done".

When I identified tasks to be done I placed them in the "Todo" column, and the tasks I was currently working on I had in the "In Progress"

I moved the story to the In Progress column when I started working on it.
When I finished working on the story, I moved it to the Done column.

Git was used for version control.


